\documentclass[10pt,journal,cspaper,compsoc]{IEEEtran}

\usepackage{cite}

\ifCLASSINFOpdf
   \usepackage[pdftex]{graphicx}
   \graphicspath{{figures/pdf/}{figures/png/}}
\else
   \usepackage[dvips]{graphicx}
   \graphicspath{{figures/eps/}}
\fi

\usepackage{multirow}
\usepackage{amssymb,amsmath,amsthm}
%\usepackage[lined,linesnumbered,boxed]{algorithm2e}
\newtheorem{theorem}{Theorem}
\newtheorem{proposition}{Proposition}
\usepackage{url}
% correct bad hyphenation here
\hyphenation{op-tical net-works}
%\hyphenation{op-tical net-works semi-conduc-tor}

\makeatletter
\newif\if@restonecol
\makeatother
\let\algorithm\relax
\let\endalgorithm\relax

\usepackage{hyperref}
\usepackage[ruled,vlined]{algorithm2e}

\begin{document}
\title{Nonobtuse remeshing with a guaranteed angle bound}

%Authors
\author{%Hui Wang% <-this % stops a space
\IEEEcompsocitemizethanks{\IEEEcompsocthanksitem ...\protect\\
% note need leading \protect in front of \\ to get a newline within \thanks as
% \\ is fragile and will error, could use \hfil\break instead.
%E-mail: see http://www.sfu.ca
\IEEEcompsocthanksitem ...}% <-this % stops a space
\thanks{}}

% The paper headers
\markboth{Prepare for TVCG}%
{Hui \MakeLowercase{\textit{et al.}}: Nonobtuse remeshing with a guaranteed angle bound}


%Abstract
\IEEEcompsoctitleabstractindextext{%
\begin{abstract}
%\boldmath
Quality meshing is an important problem in geometric modeling, visualization and scientific computing. In this paper, we propose a 3D triangular remeshing method with a \emph{guaranteed} angle bound of $[30^{\circ}, 90^{\circ}]$. Given a original 2-manifold, open or closed, a rough approximate mesh with the proposed angle bound is first generated. This is achieved by a novel extension of the classical marching cubes algorithm. Next, an iterative constrained optimization, along with constrained Laplacian smoothing, decimation, and subdivision, is performed to arrive at a close approximation of the original mesh.
\end{abstract}

% Note that keywords are not normally used for peer review papers.
\begin{keywords}
Remeshing, non-obtuse meshes, marching cube, deform-to-fit.
\end{keywords}}

\maketitle
\IEEEdisplaynotcompsoctitleabstractindextext
\IEEEpeerreviewmaketitle


%Introduction
\section{Introduction}
Efficient rendering, finite element analysis, and geometry processing such as compression, smoothing and deformation benefit form high quality meshes. Thus, a lot remeshing methods have been proposed to improve the mesh quality \cite{AUGA2008}, both in terms of geometry \cite{AVDI2003, ACDLD2003} and connectivity \cite{G2007, AYZ2012}. It is desirable for the triangular mesh to have no small angles and/or no larger angles, depending on the targeted applications \cite{S2002}. 

In this paper, we  


\subsection{Subsection Heading Here}
Subsection text here.

% needed in second column of first page if using \IEEEpubid
%\IEEEpubidadjcol

\subsubsection{Subsubsection Heading Here}
Subsubsection text here.Subsubsection text here.Subsubsection text here.Subsubsection text here.Subsubsection text here.Subsubsection text here.Subsubsection text here.Subsubsection text here.Subsubsection text here.Subsubsection text here.Subsubsection text here.Subsubsection text here.Subsubsection text here.
Subsubsection text here.Subsubsection text here.Subsubsection text here.

Subsubsection text here.Subsubsection text here.Subsubsection text here.Subsubsection text here.Subsubsection text here.Subsubsection text here.Subsubsection text here.Subsubsection text here.Subsubsection text here.Subsubsection text here.Subsubsection text here.Subsubsection text here.Subsubsection text here.Subsubsection text here.


% An example of a floating figure using the graphicx package.
% Note that \label must occur AFTER (or within) \caption.
% For figures, \caption should occur after the \includegraphics.
% Note that IEEEtran v1.7 and later has special internal code that
% is designed to preserve the operation of \label within \caption
% even when the captionsoff option is in effect. However, because
% of issues like this, it may be the safest practice to put all your
% \label just after \caption rather than within \caption{}.
%
% Reminder: the "draftcls" or "draftclsnofoot", not "draft", class
% option should be used if it is desired that the figures are to be
% displayed while in draft mode.
%
%\begin{figure}[!t]
%\centering
%\includegraphics[width=2.5in]{myfigure}
% where an .eps filename suffix will be assumed under latex,
% and a .pdf suffix will be assumed for pdflatex; or what has been declared
% via \DeclareGraphicsExtensions.
%\caption{Simulation Results}
%\label{fig_sim}
%\end{figure}

% Note that IEEE CS typically puts floats only at the top, even when this
% results in a large percentage of a column being occupied by floats.
% However, the Computer Society has been known to put floats at the bottom.


% An example of a double column floating figure using two subfigures.
% (The subfig.sty package must be loaded for this to work.)
% The subfigure \label commands are set within each subfloat command, the
% \label for the overall figure must come after \caption.
% \hfil must be used as a separator to get equal spacing.
% The subfigure.sty package works much the same way, except \subfigure is
% used instead of \subfloat.
%
%\begin{figure*}[!t]
%\centerline{\subfloat[Case I]\includegraphics[width=2.5in]{subfigcase1}%
%\label{fig_first_case}}
%\hfil
%\subfloat[Case II]{\includegraphics[width=2.5in]{subfigcase2}%
%\label{fig_second_case}}}
%\caption{Simulation results}
%\label{fig_sim}
%\end{figure*}
%
% Note that often IEEE CS papers with subfigures do not employ subfigure
% captions (using the optional argument to \subfloat), but instead will
% reference/describe all of them (a), (b), etc., within the main caption.


% An example of a floating table. Note that, for IEEE style tables, the
% \caption command should come BEFORE the table. Table text will default to
% \footnotesize as IEEE normally uses this smaller font for tables.
% The \label must come after \caption as always.
%
%\begin{table}[!t]
%% increase table row spacing, adjust to taste
%\renewcommand{\arraystretch}{1.3}
% if using array.sty, it might be a good idea to tweak the value of
% \extrarowheight as needed to properly center the text within the cells
%\caption{An Example of a Table}
%\label{table_example}
%\centering
%% Some packages, such as MDW tools, offer better commands for making tables
%% than the plain LaTeX2e tabular which is used here.
%\begin{tabular}{|c||c|}
%\hline
%One & Two\\
%\hline
%Three & Four\\
%\hline
%\end{tabular}
%\end{table}


% Note that IEEE does not put floats in the very first column - or typically
% anywhere on the first page for that matter. Also, in-text middle ("here")
% positioning is not used. Most IEEE journals use top floats exclusively.
% However, Computer Society journals sometimes do use bottom floats - bear
% this in mind when choosing appropriate optional arguments for the
% figure/table environments.
% Note that, LaTeX2e, unlike IEEE journals, places footnotes above bottom
% floats. This can be corrected via the \fnbelowfloat command of the
% stfloats package.



\section{Conclusion}
The conclusion goes here. The conclusion goes here.The conclusion goes here.The conclusion goes here.The conclusion goes here.The conclusion goes here.The conclusion goes here.The conclusion goes here.The conclusion goes here.The conclusion goes here.The conclusion goes here.The conclusion goes here.The conclusion goes here.The conclusion goes here.The conclusion goes here.The conclusion goes here.The conclusion goes here.The conclusion goes here.The conclusion goes here.The conclusion goes here. The conclusion goes here.The conclusion goes here.The conclusion goes here.The conclusion goes here.





% if have a single appendix:
%\appendix[Proof of the Zonklar Equations]
% or
%\appendix  % for no appendix heading
% do not use \section anymore after \appendix, only \section*
% is possibly needed

% use appendices with more than one appendix
% then use \section to start each appendix
% you must declare a \section before using any
% \subsection or using \label (\appendices by itself
% starts a section numbered zero.)
%


\appendices
\section{Proof of the First Zonklar Equation}
Appendix one text goes here.

% you can choose not to have a title for an appendix
% if you want by leaving the argument blank
\section{}
Appendix two text goes here.Appendix two text goes here.Appendix two text goes here.Appendix two text goes here.Appendix two text goes here.Appendix two text goes here.Appendix two text goes here.Appendix two text goes here.Appendix two text goes here.Appendix two text goes here.Appendix two text goes here.Appendix two text goes here.Appendix two text goes here.Appendix two text goes here.Appendix two text goes here.Appendix two text goes here.Appendix two text goes here.Appendix two text goes here.Appendix two text goes here.Appendix two text goes here.Appendix two text goes here.Appendix two text goes here.Appendix two text goes here.Appendix two text goes here.Appendix two text goes here.


% use section* for acknowledgement
\ifCLASSOPTIONcompsoc
  % The Computer Society usually uses the plural form
  \section*{Acknowledgments}
\else
  % regular IEEE prefers the singular form
  \section*{Acknowledgment}
\fi


The authors would like to thank...The authors would like to thank...The authors would like to thank...The authors would like to thank...The authors would like to thank...The authors would like to thank...The authors would like to thank...The authors would like to thank...The authors would like to thank...The authors would like to thank...The authors would like to thank...The authors would like to thank...The authors would like to thank...


% Can use something like this to put references on a page
% by themselves when using endfloat and the captionsoff option.
\ifCLASSOPTIONcaptionsoff
  \newpage
\fi



%Reference
\bibliographystyle{IEEEtran}
\bibliography{tvcg}
\end{document}



